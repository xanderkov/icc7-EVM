
\chapter*{Исследования расслоения динамической памяти}


\img{60mm}{task1.jpg}{Результат исследования расслоения динамической памяти}
 

\section*{Вывод}
 Оперативная память расслоена и неоднородна, поэтому обращение к последовательно расположенным данным требует различного времени из-за наличия открытыя и закрытыя  страниц динамической памяти. При этом чем больше адресное расстояние, тем больше время доступа. В связи с этим для создания эффективных программ необходимо учитывать расслоение памяти и размещать рядом данные для непосредственной обработки.



\chapter*{Сравнение эффективности ссылочных и векторных структур}

\img{60mm}{task2.jpeg}{Результат исследования расслоения динамической памяти}


\section*{Вывод}
 Видна проблема семантического разрыва: машина не присоблена к работе со ссылочными структурами. Использовать структуры данных надо с учётом скрытых технологических констант. Если алгоритм предполагает возможность использования массива, а списки не дают существенной разницы, то использование массива вполне оправдано.



\chapter*{Исследование эффективности программной предвыборки}

Исходные данные: степень ассоциативности и размер TLB данных.


\img{60mm}{task3.jpeg}{Результат исследования расслоения динамической памяти}



\section*{Вывод}
 Обработка больших массивов информации сопряжена с открытием большого количества физических страниц памяти. При первом обращении к странице памяти наблюдается увеличенное время доступа к данным в 20 раз, так как оно при отсутствии информации в TLB вызывает двойное обращение к оперативной памяти: сначала за информацией из таблицы страниц, а далее за востребованными данными. Поэтому для ускорения работы программы можно использовать предвыборку. Например, пока процессор занят некоторыми расчетами и не обращается к памяти, можно заблаговременно провести все указанные действия благодаря дополнительному запросу небольшого количества данных из оперативной памяти. 

Также стоит стараться не использовать в программе массивы, к которым обращение выполняется только один раз.


\chapter*{Исследование способов эффективного чтения оперативной памяти}



\img{60mm}{task4.jpeg}{Результат исследования расслоения динамической памяти}

\section*{Вывод}
Эффективная обработка нескольких векторных структур данных без их дополнительной оптимизации не использует в должной степени возможности аппаратных ресурсов. 

Для создания структур данных, оптимизирующих их обработку, необходимо передавать в каждом пакете только востребованную для вычислений информацию.То есть для ускорения алгоритмов необходимо правильно упорядочивать данные.


\chapter*{Исследование конфликтов в кэш-памяти}



\img{60mm}{task5.jpeg}{Результат исследования расслоения динамической памяти}


\section*{Вывод}
Попытка читать данные из оперативной памяти с шагом, кратным размеру банка, приводит к их помещению в один и тот же набор. Если же количество запросов превосходит степень ассоциативности кэш-памяти, т.е. количество банков или количество линеек в наборе, то наблюдается постоянное вытеснение данных из кэш-памяти, причем больший ее объем остается незадействованным. Кэш память ускоряет работу процессора в 6.7 раз


\chapter*{Сравнение алгоритмов сортировки}


\img{60mm}{task6.jpeg}{Результат исследования расслоения динамической памяти}


\section*{Вывод}
 Существует алгоритм сортировки менее чем линейной вычислительной сложности.

 
