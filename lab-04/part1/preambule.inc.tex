\usepackage{cmap}
\usepackage[T1]{fontenc} 
\usepackage[utf8]{inputenc}
\usepackage[english,russian]{babel}

\usepackage{caption}
\usepackage{subcaption}

\usepackage{float}

\usepackage{enumitem}

\usepackage{graphicx}
\usepackage{multirow}


\usepackage{pgfplots}
\pgfplotsset{compat=newest}
\usepgfplotslibrary{units}

\usepackage{caption}
\captionsetup{labelsep=endash}
\captionsetup[figure]{name={Рисунок}}
\captionsetup[subtable]{labelformat=simple}
\captionsetup[subfigure]{labelformat=simple}
\renewcommand{\thesubtable}{\text{Таблица }\arabic{chapter}\text{.}\arabic{table}\text{.}\arabic{subtable}\text{ --}}
\renewcommand{\thesubfigure}{\text{Рисунок }\arabic{chapter}\text{.}\arabic{figure}\text{.}\arabic{subfigure}\text{ --}}


\usepackage{textcomp}

\usepackage{amsmath}
\usepackage{amsfonts}
\usepackage{array}

\usepackage{geometry}
\geometry{left=30mm}
\geometry{right=15mm}
\geometry{top=20mm}
\geometry{bottom=20mm}
\geometry{foot=1.7cm}

\usepackage{titlesec}
\titleformat{\section}
{\normalsize\bfseries}
{\thesection}
{1em}{}
\titlespacing*{\chapter}{0pt}{-30pt}{8pt}
\titlespacing*{\section}{\parindent}{*4}{*4}
\titlespacing*{\subsection}{\parindent}{*4}{*4}

% Маркировка для списков
\def\labelitemi{$\circ$}
\def\labelitemii{$*$}


\usepackage{setspace}
\onehalfspacing % Полуторный интервал

\frenchspacing
\usepackage{indentfirst} % Красная строка

\usepackage{titlesec}
\usepackage{xcolor}
% Названия глав
\titleformat{\section}{\normalsize\textmd}{\thesection}{1em}{}

\definecolor{gray35}{gray}{0.35}

\newcommand{\hsp}{\hspace{20pt}} % длина линии в 20pt

\titleformat{\chapter}[hang]{\Huge}{\textcolor{gray35}{\thechapter.}\hsp}{0pt}{\Huge\textmd}

\titleformat{\section}{\Large}{\textcolor{gray35}\thesection}{20pt}{\Large\textmd}
\titleformat{\subsection}{\Large}{\thesubsection}{20pt}{\Large\textmd}
\titleformat{\subsubsection}{\normalfont\textmd}{}{0pt}{}

% Настройки введения

\addtocontents{toc}{\setcounter{tocdepth}{2}}
\addtocontents{toc}{\setcounter{secnumdepth}{1}}

\usepackage{tocloft,lipsum,pgffor}

\addtocontents{toc}{~\hfill\textnormal{Страница}\par}

\renewcommand{\cftpartfont}{\normalfont\textmd}

\addto\captionsrussian{\renewcommand{\contentsname}{Содержание}}
\renewcommand{\cfttoctitlefont}{\Huge\textmd}

\renewcommand{\cftchapfont}{\normalfont\normalsize}
\renewcommand{\cftsecfont}{\normalfont\normalsize}
\renewcommand{\cftsubsecfont}{\normalfont\normalsize}
\renewcommand{\cftsubsubsecfont}{\normalfont\normalsize}

\renewcommand{\cftchapleader}{\cftdotfill{\cftdotsep}}

\usepackage{listings}
\usepackage{xcolor}

\usepackage[pdftex]{hyperref} % Гиперссылки
\hypersetup{hidelinks}

% Листинги 
\usepackage{listings}

\definecolor{darkgray}{gray}{0.15}

\definecolor{teal}{rgb}{0.25,0.88,0.73}
\definecolor{gray}{rgb}{0.5,0.5,0.5}
\definecolor{b-red}{rgb}{0.88,0.25,0.41}
\definecolor{royal-blue}{rgb}{0.25,0.41,0.88}

\usepackage{listings}
\lstset{
	aboveskip=3mm,
	belowskip=3mm,
	frame=tb,
	frame=single,
	basicstyle=\footnotesize\ttfamily,
	numberstyle=\tiny\color{gray},
	keywordstyle=\color{royal-blue},
	commentstyle=\color{gray35},
	stringstyle=\color{b-red},
	numbers=left,
	numbersep=5pt,
	numberstyle=\tiny,
	showstringspaces=false, 
	captionpos=t,
	tabsize=4,
	language=C
}

% какой то сложный кусок со стак эксчейндж для квадратных скобок
\makeatletter
\newenvironment{sqcases}{%
	\matrix@check\sqcases\env@sqcases
}{%
	\endarray\right.%
}
\def\env@sqcases{%
	\let\@ifnextchar\new@ifnextchar
	\left\lbrack
	\def\arraystretch{1.2}%
	\array{@{}l@{\quad}l@{}}%
}
\makeatother

% и для матриц
\makeatletter
\renewcommand*\env@matrix[1][\arraystretch]{%
	\edef\arraystretch{#1}%
	\hskip -\arraycolsep
	\let\@ifnextchar\new@ifnextchar
	\array{*\c@MaxMatrixCols c}}
\makeatother
